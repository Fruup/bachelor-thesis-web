% !TEX root = ../my-thesis.tex
%
\chapter{Future work}
\label{sec:futurework}

\section{Performance}

Our focus was implementing a working version of the proposed algorithms. Thus, there is a lot of room for improvement of the program's performance. The main source of performance gains will be employing the GPU. Almost every aspect of the algorithm can be implemented in some kind of shader program that the GPU can run in parallel thousands of times: computation of $\textbf{G}_i$ and ray marching iterations to name just a few. It is even possible to execute parts of the pipeline in parallel: Because no neighborhood search has to be performed in the depth pass, it can be executed in tandem with the pre-processing step. \textit{Vulkan} enables a lot of customization, especially for the synchronization of parallel work, making it a natural fit for this application.

\section{Data structure}

Currently, we use multiple separate data structures for neighborhood search and the density grid. It may be beneficial to design a single structure capable of all these features. If the underlying simulation is compatible, simulation and visualization could even share the same structures so they do not have to be rebuilt multiple times per frame.

\section{Multisampling}

The images generated by our algorithm show signs of aliasing. There is a harsh edge between pixels belonging to the fluid and pixels belonging to the background. In computer graphics, a common method of mitigating this is called multi-sampling. For each pixel, multiple samples of the scene are recorded in close proximity and averaged into a single value. This can be achieved by rendering the scene to an image with higher dimensions than the screen and then downsampling it to fit. In our case, this would only have to be done to the composition image. The position and normal buffer could be linearly interpolated between pixels, keeping the ray marching phase unchanged.

\section{Additional fluid features}

Fluids can have many more features than merely color, opacity, reflection et cetera. For example, Ihmsen et al. \cite{Ihmsen:2012} propose a method for displaying foam and bubbles, contributing to the realism of the visualization.
Another crucial part of rendering is shadows. This is prominently achieved with a technique called shadow mapping. The scene is rendered from the perspective of a source of light. For this step, a more lightweight algorithm can be used: Instead of ray marching to find the exact surface position, a smoothed version of the primary depth buffer can be used as a good approximation.
Green \cite{Green:2010} suggests to render the thickness of the fluid to a texture which can be used during image composition to assign dense regions a more opaque color. This models the physical property of molecules scattering light away from the viewers eyes while it travels through a medium.

